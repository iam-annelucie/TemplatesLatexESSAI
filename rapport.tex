%% @Author: Ines Abdeljaoued Tej
%  @Date:   2018-06
%% @Class:  PFE de l'ESSAI - Universite de Carthage, Tunisie.

\documentclass[a4paper, oneside, french, 12pt, final]{extreport}
\usepackage{graphicx}

\parindent 0cm
\usepackage{makeidx}
\makeindex

\usepackage[lined,boxed,commentsnumbered, ruled,vlined,linesnumbered]{algorithm2e}
\usepackage{amsthm}
\newtheorem{theorem}{Theorem}[chapter]
\newtheorem{definition}{Definition}[chapter]
\newtheorem{exemple}{Example}[chapter]


%\usepackage[nottoc]{tocbibind}
%\addcontentsline{toc}{section}{References}

\providecommand{\keywords}[1]{\textbf{\textit{Keywords---}} #1}

\usepackage{etoolbox}
%\makeatletter
%\patchcmd{\thebibliography}{%
%  \chapter*{\bibname}\@mkboth{\MakeUppercase\bibname}%{\MakeUppercase\bibname}}{%
%  \section{References}}{}{}
%\makeatother

\usepackage[utf8]{inputenc}
\usepackage[english]{babel}

\usepackage[nottoc]{tocbibind}

\textwidth 18cm
\textheight 24cm
\topmargin -0.5cm
\oddsidemargin -1cm

% set font encoding for PDFLaTeX or XeLaTeX
\usepackage{ifxetex}
\ifxetex
  \usepackage{fontspec}
\else
  \usepackage[T1]{fontenc}
  \usepackage[utf8]{inputenc}
  \usepackage{lmodern}
\fi


% Enable SageTeX to run SageMath code right inside this LaTeX file.
% documentation: http://mirrors.ctan.org/macros/latex/contrib/sagetex/sagetexpackage.pdf
%\usepackage{sagetex}


\newcommand{\reportTitle} {%
  %\textsc{Graduation Project Report}
  \textsc{Projet de Fin d'\'etudes}
}

\newcommand{\reportAuthor} {%
  FirstName \textsc{LastName}%
}

\newcommand{\reportSubject} {%
  My very attractive \\ Title%
}

\newcommand{\dateSoutenance} {%
  12/06/2018%
}

\newcommand{\studyDepartment} {%
  Entreprise d'accueil %Statistique
}

\newcommand{\ESSAI} {%
  %Higher School of Statistics and Information Analysis
  Ecole Sup\'erieure de la Statistique et de l'Analyse de l'Information
}

%\newcommand{\codePFE} {% Reference
%  Code PFE%
%}

\newcommand{\juryPresident} {%
  Mr Ben Foulen \textsc{Foulenia}%
}
\newcommand{\juryPresidentDesc} {%
  President%
}

\newcommand{\juryMemberOne} {%
  Ms Ben Foulena \textsc{Foulen}%
}
\newcommand{\juryMemberOneDesc} {%
  Examiner %Mentor
}

\newcommand{\juryMemberTwo} {%
  Mr Ben Foulen \textsc{Fouleni}%
}
\newcommand{\juryMemberTwoDesc} {%
  Reviewer% Examiner, Reporter
}

\newcommand{\juryMemberThree} {%
	M. Ben Foulen \textsc{Fouleni}%
}
\newcommand{\juryMemberThreeDesc} {%
	Supervisor% Examiner, Reporter
}

\newcommand{\juryMemberFour} {%
	M. Ben Foulen \textsc{Fouleni}%
}
\newcommand{\juryMemberFourDesc} {%
	Mentor% Examiner, Reporter
}


\newcommand{\specialcell}[1]{%
  \begin{tabularx}{\textwidth}{@{}X@{}}#1\end{tabularx}%
}

%%%%%%%%%%%%%%%%%%%%%%%%%%%%%%%%%%%%%%%%%%%%%%%%%%%%%%%
% Add your own commands here
%%%%%%%%%%%%%%%%%%%%%%%%%%%%%%%%%%%%%%%%%%%%%%%%%%%%%%%
\newcommand{\MyCommand} {%
  Does nothing really%
}


% used in maketitle
\title{\reportSubject}
\author{\reportAuthor}

% Enable SageTeX to run SageMath code right inside this LaTeX file.
% documentation: http://mirrors.ctan.org/macros/latex/contrib/sagetex/sagetexpackage.pdf
%\usepackage{sagetex}

%\hypersetup{
%  pdftitle={\reportTitle~-~\reportSubject},%
%  pdfauthor={\reportAuthor},%
%  pdfsubject={\reportSubject},%
%  pdfkeywords={report} {internship} {pfe} {enis}
%}

\usepackage{graphics}
\usepackage{graphicx}


\usepackage[acronym,toc,section=chapter]{glossaries}
\makeglossaries

\newacronym{abc}{ABC}{A contrived acronym}
\newacronym{efg}{EFG}{Another acronym}
\newacronym{svm}{SVM}{Support Vector Machines}

\pagenumbering{roman} 
\begin{document}
\thispagestyle{empty}
\begin{titlepage}
\begin{center}


%%%%%%%%%%%%%%%%%%%%%%%%%%%%%%%%%%%%%%%%%%%%%%%
% THE HEADER
%%%%%%%%%%%%%%%%%%%%%%%%%%%%%%%%%%%%%%%%%%%%%%%

\includegraphics[scale=0.15]{embleme.jpg}
\vspace{0.5cm}

{%
  \fontsize{9pt}{9pt}\selectfont%
  \begin{tabular}{c}
    R\'epublique Tunisienne \\
    Minist\`ere de l'Enseignement Supérieur et de la Recherche Scientifique \\%
    Universit\'e de Carthage - \ESSAI{}\\ 
  \end{tabular}
}

\vspace{0.5cm}

\includegraphics[scale=0.04]{universite-carthage.jpg}


%%%%%%%%%%%%%%%%%%%%%%%%%%%%%%%%%%%%%%%%%%%%%%%
% THE PAGE CONTENT
%%%%%%%%%%%%%%%%%%%%%%%%%%%%%%%%%%%%%%%%%%%%%%%

\vspace{5pt} {%
  \renewcommand*{\familydefault}{\defaultFont}
  \fontsize{46pt}{46pt}\selectfont%
  % MEMOIRE\\%
  %\reportTitle{}%\\\textsc{Report}\\%
}

%\vspace{5pt}

\vspace{10pt}
{\textit{Rapport de Projet de Fin d'Etudes soumis afin d'obtenir le}}\\

\vspace{10pt}
{\textbf{\large Diplôme National d'Ingénieur en Statistique et Analyse de l'Information}}\\

\includegraphics[scale=0.4]{logo-essai.jpg}\\

\vspace{5pt}
\textbf{\textit{Réalisé par}}\\
\vspace{10pt} {%
  \fontsize{14pt}{14pt}\selectfont%
  {\bfseries\Large\sc \reportAuthor}\\
}%

\vspace{5pt} {%
  \renewcommand*{\familydefault}{\defaultFont}
  \fontsize{27pt}{27pt}\selectfont%
  \rule{0.5\textwidth}{.4pt}\\
  \vspace{10pt}
  \reportSubject{}\\%
  \vspace{10pt}
  \rule{0.5\textwidth}{.4pt}
}

\vspace{5pt}
Soutenu le\, \dateSoutenance\,\, devant le Jury compos\'e de :\\
%Soutenu le \dateSoutenance, devant la commission d'examen:\\
\vspace{10pt}
\begin{tabular}{p{0.3\linewidth} p{0.15\linewidth}}
  \juryPresident{} & \juryPresidentDesc{}\\
  \juryMemberOne{} & \juryMemberOneDesc{}\\
  \juryMemberTwo{} & \juryMemberTwoDesc{}\\
  \juryMemberThree{} & \juryMemberThreeDesc{}\\
%  \juryMemberFour{} & \juryMemberFourDesc{}\\
\end{tabular}

%\vfill

\vspace{10pt}%
\textbf{\textit{Projet de Fin d'Etudes fait \`a}}\\
\vspace{5pt}
(\studyDepartment)\\
%\includegraphics[scale=0.4]{logo-studyDepartment.jpg}
\end{center}
\end{titlepage}

% ###############################
% # HELP COMMANDS               #
% ###############################
%
% -1 \part{part}
%  0 \chapter{chapter}
%  1 \section{section}
%  2 \subsection{subsection}
%  3 \subsubsection{subsubsection}
%  4 \paragraph{paragraph}
%  5 \subparagraph{subparagraph}


%%%%%%%%%%%%%%%%%%%%%%%%%%%%%%%%%%%%%%%%%%%%%%%%%%%%%%%
% Dédicace et Remerciements
%%%%%%%%%%%%%%%%%%%%%%%%%%%%%%%%%%%%%%%%%%%%%%%%%%%%%%%

%\chapter*{Dedication}
\chapter*{D\'edicace}
%\addcontentsline{toc}{chapter}{Dedication}
\thispagestyle{empty}
%
%For all they have endured to satisfy all my needs and wishes

\begin{center}
{\it 
	
A ... pour son(leur) sacrifice et son(leur) soutien, \\
en témoignage de mon infinie reconnaissance et mon profond attachement \\
\vspace{1cm}
A tous ceux qui me sont chers...

}
\end{center}
%
%\nopagebreak{%
% And maybe a quote here
% \raggedright\hspace{5.75cm} To all of you,~\\
%\raggedright\hspace{7.75cm} I dedicate this work.
%  \raggedleft\normalfont\large\itshape{} \reportAuthor\par%
%}
%
%\cleardoublepage%

%\chapter*{Thanks}
\chapter*{Remerciements}
%\addcontentsline{toc}{chapter}{Thanks}
\thispagestyle{empty}
%
%Au terme de ce travail (A l'issue de ce travail), je tiens à remercier M., Mme, Pr., Dr. pour sa disponibilité et ses conseils judicieux. \\

Je n'aurais jamais pu réaliser ce projet sans la précieuse aide et sans le soutien d'un grand nombre de personnes dont la générosité, la bonne humeur et l'intérêt manifestés à l'égard de mon PFE m'ont permis de progresser. \\

%En premier lieu, je tiens à remercier mon encadrant universitaire, \juryMemberFour{}, pour la confiance qu'il m'a accordée en acceptant d'encadrer ce travail, pour ses multiples conseils et pour toutes les heures qu'il a consacrées à diriger ce travail. \\ 

%Je souhaiterais exprimer ma gratitude à \juryMemberThree{}, pour m’avoir donné envie de réaliser un mémoire sur ... au sein de \og \studyDepartment \fg. Je le remercie également pour son accueil chaleureux à chaque fois que j'ai sollicité son aide, ainsi que pour ses multiples encouragements. J’ai été extrêmement sensible à ses qualités humaines d'écoute et de compréhension tout au long de ce travail de mémoire. \\

%J'aimerais également dire à \juryPresident{} à quel point je suis honorée pour avoir accepté de présider ce jury de PFE. Je suis infiniment gré à  \juryMemberOne{} de s’être rendu disponible et d’avoir accepté la fonction de rapporteur. De même, je suis particulièrement reconnaissant(e) à (\juryMemberTwo{} de l'intérêt qu'il/elle a manifesté à l'égard de ce projet en s'engageant à être rapporteur. \\

Ma reconnaissance va à ceux qui ont plus particulièrement assuré le soutien affectif de ce travail : ma famille ainsi que mes amis. Mes parents... 



%%%%%%%%%%%%%%%%%%%%%%%%%%%%%%%%%%%%%%%%%%%%%%%%%%%%%%%
% Divers chapitres
%%%%%%%%%%%%%%%%%%%%%%%%%%%%%%%%%%%%%%%%%%%%%%%%%%%%%%%

\tableofcontents
%\addcontentsline{toc}{chapter}{\contentsname}

\listoffigures
%\addcontentsline{toc}{chapter}{Liste des Figures}
\listoftables
%\addcontentsline{toc}{chapter}{Liste des Tableaux}
\listofalgorithms
\addcontentsline{toc}{chapter}{Liste des algorithmes}

\newpage
\pagenumbering{arabic}
\chapter*{Introduction}
\label{chap:general_intorduction}
\input{introduction.tex}


\chapter{Données étudiés}%
\label{chap:chapterone}
%% @Author: Ines Abdeljaoued Tej
%  @Date:   2018-06
%% @Class:  PFE de l'ESSAI - Universite de Carthage, Tunisie.

%%%%%%%%%%%%%%%%%%%%%%%%%%%%
% SECTION                  %
%%%%%%%%%%%%%%%%%%%%%%%%%%%%

Insérer un résumé du chapitre : dans la Section \ref{chap:sectionone} nous introduisons la problématique...

\section{Section une}
\label{chap:sectionone}

\subsection{Sub section One}

And your chapter one goes here \cite{web001,Nom2012}. \\
  Lorem ipsum dolor sit amet, consectetur adipisicing elit, sed do eiusmod
  tempor incididunt ut labore et dolore magna aliqua. Ut enim ad minim veniam, quis nostrud exercitation ullamco laboris nisi ut aliquip ex ea commodo consequat. Duis aute irure dolor in reprehenderit in voluptate velit esse \cite{Bird02nltk:the}
  cillum dolore eu fugiat nulla pariatur. Excepteur sint occaecat cupidatat non
  proident, sunt in culpa qui officia deserunt mollit anim id est laborum.

  \begin{figure}[h]%
    \center%
    \includegraphics[width=0.3\textwidth]{diamonds.pdf}
    \caption[This is a test image]{Test Image}\label{fig:test}%
  \end{figure}


\begin{table}\begin{center}
\begin{tabular}{c|c}
Entrée & Sortie \\ \hline 
A & B \\
C & D
\end{tabular}
\caption{Test Table}\end{center}
\end{table}


\subsection{Sub section Two}

  This is a second subsection\cite{gen1972}, \cite{schaeffer99}. ~\\
  Lorem ipsum dolor sit amet, consectetur adipisicing elit, sed do eiusmod
  tempor incididunt ut labore et dolore magna aliqua. Ut enim ad minim veniam,
  quis nostrud exercitation ullamco laboris nisi ut aliquip ex ea commodo
  consequat. Duis aute irure dolor in reprehenderit in voluptate velit esse
  cillum dolore eu fugiat nulla pariatur. Excepteur sint occaecat cupidatat non
  proident, sunt in culpa qui officia deserunt mollit anim id est laborum.

  \begin{description}\addtolength{\itemsep}{-0.35\baselineskip}%
    \item[\textbullet~\bfseries Menu Item] \hfill \\%
      Menu Description.~\\%
      {\textbf{Focus topics:~}\emph{Topic one, topic two, topic three, ...}}%
    %
    \item[\textbullet~\bfseries Menu Item] \hfill \\%
      Menu Description.~\\%
      {\textbf{Focus topics:~}\emph{Topic one, topic two, topic three, ...}}%
    %
    \item[\textbullet~\bfseries Menu Item] \hfill \\%
      Menu Description.~\\%
      {\textbf{Focus topics:~}\emph{Topic one, topic two, topic three, ...}}%
  \end{description}

  Also bullets such as:%
  \begin{itemize}\addtolength{\itemsep}{-0.35\baselineskip}%
    \item One%
    \item Two%
    \item Three%
    \item Four%
    \item \ldots%
  \end{itemize}%
  %
\section{powers series} \label{subsection}

\begin{equation} \label{eq:1}
\sum_{i=0}^{\infty} a_i x^i
\end{equation}

The equation \ref{eq:1} is a typical power series.

\section*{Conclusion}

Insérer une brève conclusion du chapitre. 


\chapter{Modèles utilisés et Applications}
\label{chap:2}
\input{chapitre2.tex}

\chapter{Indicateurs de performances et Résultats}
\label{chap:3}
\input{chapitre3.tex}

\chapter*{Conclusion et Perspectives}
\label{chap:conclusion}
\markboth{\MakeUppercase{Conclusion}}{}%
\addcontentsline{toc}{chapter}{Conclusion}
  And a very interesting conclusion here\@. ~\\
  Lorem ipsum dolor sit amet, consectetur adipisicing elit, sed do eiusmod
  tempor incididunt ut labore et dolore magna aliqua. Ut enim ad minim veniam,
  quis nostrud exercitation ullamco laboris nisi ut aliquip ex ea commodo
  consequat.

\newpage
\appendix
\addcontentsline{toc}{chapter}{Annexes}
%\markboth{\MakeUppercase{Annexe}}{}

\chapter{Code R pour résoudre la problématique}
\label{chap:appendix}


\section{Pré-traitement des données}
\section{Code R pour les modèles}

 An appedix if you need it.
 
 \begin{verbatim}
 Insérer ici le code !
 \end{verbatim}

\section{Librairies utilisées}
 
  Lorem ipsum dolor sit amet, consectetur adipisicing elit, sed do eiusmod
  tempor incididunt ut labore et dolore magna aliqua. Ut enim ad minim veniam,
  quis nostrud exercitation ullamco laboris nisi ut aliquip ex ea commodo.


%%%%%%%%%%%%%%%%%%%%%%%%%%%%%%%%%%%%%%%%%%%%%%%%%%%%
% Don't touch this, it is auto generated
%%%%%%%%%%%%%%%%%%%%%%%%%%%%%%%%%%%%%%%%%%%%%%%%%%%%
\nocite{*}

%\phantomsection{}
%\addcontentsline{toc}{chapter}{Webography}
%\printbibliography[title={Webography},type=online]

%\phantomsection{}
%\addcontentsline{toc}{chapter}{Bibliography}
%\printbibliography[title={Bibliography},nottype=online]

%\printbibheading %exemple de bibliographie divisée en sections. Pour ajouter des oeuvres non citées,utiliser \nocite

%\printbibliography[keyword=pratique,heading=subbibliography,title={Théories littéraires dans les jeux vidéo}]
%\printbibliography[keyword=litteraire,heading=subbibliography,title={Narratologie et structuralisme}]

%\printbibliography[keyword=jeu,heading=subbibliography,title={\emph{Games studies}}]

\bibliographystyle{apalike}
\bibliography{Biblio.bib}

\cleardoublepage%

\addtocontents{toc}{\protect\setcounter{tocdepth}{3}}

\printglossaries
\printindex

\newpage
\begin{abstract}
	Insérer un résumé en Français \\
	
	\keywords{Ici, Mettre, Cinq, Mots, Clés.}
\end{abstract}


\end{document}
